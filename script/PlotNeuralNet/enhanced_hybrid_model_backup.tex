% Enhanced Hybrid Model Visualization - 突出复数处理和残差连接
% 改进特点: 1) 清晰的复数/实数域分离  2) 强化残差连接可视化
%          3) 详细的数据流标注      4) 模块功能细节展示

\documentclass[border=8pt, multi, tikz]{standalone}
\usepackage{import}
\subimport{./layers/}{init}
\usetikzlibrary{positioning}
\usetikzlibrary{3d}
\usetikzlibrary{patterns}
\usetikzlibrary{decorations.pathreplacing}

% 增强颜色方案 - 区分不同处理域
\def\ComplexDomainColor{rgb:cyan,6;blue,4;white,2}   % 复数域 - 青蓝色
\def\RealDomainColor{rgb:orange,6;red,3;white,2}     % 实数域 - 橙色
\def\ResidualColor{rgb:purple,7;blue,3;white,2}      % 残差连接 - 紫色
\def\PoolColor{rgb:red,5;orange,3;white,2}           % 池化层 - 红橙色
\def\TransitionColor{rgb:green,5;yellow,3;white,2}   % 过渡层 - 绿黄色

% 定义域标识符 (避免Unicode问题)
\newcommand{\complexdomain}[1]{\textcolor{cyan!80!black}{\textbf{[C]}} #1}
\newcommand{\realdomain}[1]{\textcolor{orange!80!black}{\textbf{[R]}} #1}

\begin{document}
\begin{tikzpicture}

% 定义连接样式
\tikzstyle{complex_flow}=[ultra thick,draw=cyan!80!black,opacity=0.8]
\tikzstyle{real_flow}=[ultra thick,draw=orange!80!black,opacity=0.8]
\tikzstyle{residual_flow}=[very thick,draw=purple!80!black,opacity=0.9,dashed]
\tikzstyle{attention_flow}=[thick,draw=red!70!black,opacity=0.7,dotted]

% 背景域区分 (进一步淡化,减小区域)
\fill[cyan!5!white,opacity=0.3] (-1,-1.5) rectangle (19,1.5);
\fill[orange!5!white,opacity=0.3] (19,-1.5) rectangle (31,1.5);

% 域标签 (更简洁的位置)
\node[above] at (9,2.2) {\normalsize\textbf{Complex Domain}};
\node[above] at (25,2.2) {\normalsize\textbf{Real Domain}};

% ========== 输入模块 ==========
\pic[shift={(0,0,0)}] at (0,0,0) {Box={
    name=input_iq,
    caption=I/Q Input,
    fill=\ComplexDomainColor,
    height=20,
    width=3,
    depth=50
}};

% ========== 复数重构模块 ==========
\pic[shift={(2.5,0,0)}] at (0,0,0) {Box={
    name=complex_formation,
    caption=Complex Form,
    fill=\ComplexDomainColor,
    height=50,
    width=4,
    depth=20
}};

% ========== 复数卷积特征提取 ==========
\pic[shift={(5.5,0,0)}] at (0,0,0) {Box={
    name=complex_conv1d,
    caption=Complex Conv1D,
    fill=\ComplexDomainColor,
    height=48,
    width=6,
    depth=18
}};

\pic[shift={(8,0,0)}] at (0,0,0) {Box={
    name=complex_batchnorm,
    caption=Complex BN,
    fill=\ComplexDomainColor,
    height=48,
    width=5,
    depth=16
}};

\pic[shift={(10.5,0,0)}] at (0,0,0) {Box={
    name=complex_activation,
    caption=Complex ReLU,
    fill=\ComplexDomainColor,
    height=48,
    width=5,
    depth=14
}};

% ========== 复数池化 ==========
\pic[shift={(13,0,0)}] at (0,0,0) {Box={
    name=complex_maxpool,
    caption=Complex MaxPool,
    fill=\PoolColor,
    height=24,
    width=5,
    depth=12
}};

% ========== 残差学习模块 ==========
% ResBlock 1 主路径
\pic[shift={(16,0,0)}] at (0,0,0) {Box={
    name=resblock1_main,
    caption=ResBlock-1,
    fill=\ResidualColor,
    height=24,
    width=7,
    depth=10
}};

% ResBlock 1 跳跃连接
\pic[shift={(16,2,0)}] at (0,0,0) {Box={
    name=resblock1_skip,
    caption=Skip,
    fill=\ResidualColor,
    height=8,
    width=7,
    depth=4
}};

% ResBlock 2 主路径
\pic[shift={(18.5,0,0)}] at (0,0,0) {Box={
    name=resblock2_main,
    caption=ResBlock-2,
    fill=\ResidualColor,
    height=12,
    width=8,
    depth=8
}};

% ResBlock 2 跳跃连接
\pic[shift={(18.5,2,0)}] at (0,0,0) {Box={
    name=resblock2_skip,
    caption=Skip,
    fill=\ResidualColor,
    height=6,
    width=8,
    depth=4
}};

% ========== 域转换模块 ==========
\pic[shift={(21.5,0,0)}] at (0,0,0) {Box={
    name=global_avg_pool,
    caption=Global Avg Pool,
    fill=\TransitionColor,
    height=4,
    width=8,
    depth=6
}};

\pic[shift={(24,0,0)}] at (0,0,0) {Box={
    name=magnitude_extraction,
    caption=Complex to Real,
    fill=\TransitionColor,
    height=3,
    width=7,
    depth=5
}};

% ========== 实数域分类 ==========
\pic[shift={(27,0,0)}] at (0,0,0) {Box={
    name=dense_classifier,
    caption=Dense 256,
    fill=\RealDomainColor,
    height=2,
    width=6,
    depth=4
}};

\pic[shift={(29.5,0,0)}] at (0,0,0) {Box={
    name=output_softmax,
    caption=Output,
    fill=\RealDomainColor,
    height=1,
    width=4,
    depth=3
}};

% ========== 主要数据流连接 (简化标签) ==========
\draw [complex_flow] (input_iq-east) -- (complex_formation-west);
\draw [complex_flow] (complex_formation-east) -- (complex_conv1d-west);
\draw [complex_flow] (complex_conv1d-east) -- (complex_batchnorm-west);
\draw [complex_flow] (complex_batchnorm-east) -- (complex_activation-west);
\draw [complex_flow] (complex_activation-east) -- (complex_maxpool-west);
\draw [complex_flow] (complex_maxpool-east) -- (resblock1_main-west);
\draw [complex_flow] (resblock1_main-east) -- (resblock2_main-west);
\draw [complex_flow] (resblock2_main-east) -- (global_avg_pool-west);
\draw [complex_flow] (global_avg_pool-east) -- (magnitude_extraction-west);

% 域转换连接
\draw [attention_flow] (magnitude_extraction-east) -- (dense_classifier-west);
\draw [real_flow] (dense_classifier-east) -- (output_softmax-west);

% ========== 残差连接 ==========
% ResBlock 1 残差连接
\draw [residual_flow] (complex_maxpool-north) to[out=90,in=180] (resblock1_skip-west);
\draw [residual_flow] (resblock1_skip-south) to[out=270,in=90] (resblock1_main-north);

% ResBlock 2 残差连接  
\draw [residual_flow] (resblock1_main-north) to[out=90,in=180] (resblock2_skip-west);
\draw [residual_flow] (resblock2_skip-south) to[out=270,in=90] (resblock2_main-north);

% ========== 标题和注释 (简化并重新定位) ==========
\node[above] at (15,4) {\Large\textbf{Enhanced Hybrid RF Signal Classifier}};

% 简化的性能指标 (移到更远位置)
\node[above] at (25,3) {\footnotesize\textbf{1.3M params | 65.38\% accuracy}};

% 简化的输入输出标识
\node[below] at (0,-2) {\tiny\textbf{I/Q}};
\node[below] at (29.5,-2) {\tiny\textbf{11 Classes}};

\end{tikzpicture}
\end{document}